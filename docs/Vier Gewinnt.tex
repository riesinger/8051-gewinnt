\documentclass[
	fontsize=12pt,
	paper=A4,
	twoside=false,
	listof=totoc,            % Tabellen- und Abbildungsverzeichnis ins Inhaltsverzeichnis
	bibliography=totoc,      % Literaturverzeichnis ins Inhaltsverzeichnis aufnehmen
	titlepage,               % Titlepage-Umgebung anstatt \maketitle
	headsepline,             % horizontale Linie unter Kolumnentitel
	abstracton,              % Überschrift einschalten, Abstract muss in {abstract}-Umgebung stehen
]{scrreprt}                  % Verwendung von KOMA-Report
\usepackage[utf8]{inputenc}  % UTF8 Encoding einschalten
\usepackage[ngerman]{babel}  % Neue deutsche Rechtschreibung
\usepackage[T1]{fontenc}     % Ausgabe von westeuropäischen Zeichen (auch Umlaute)
\usepackage{graphicx}        % Einbinden von Grafiken erlauben
\usepackage[onehalfspacing]{setspace}       % Zeilenabstand \singlespacing, \onehalfspaceing, \doublespacing
\usepackage[
	%showframe,                % Ränder anzeigen lassen
	left=2.7cm, right=2.5cm,
	top=2.5cm,  bottom=2.5cm,
	includeheadfoot
]{geometry}                      % Seitenlayout einstellen
\usepackage{mathpazo}            % Einstellung der verwendeten Schriftarten
\usepackage{scrpage2}            % Gestaltung von Fuß- und Kopfzeilen
\usepackage{acronym}             % Abkürzungen, Abkürzungsverzeichnis
\usepackage{titletoc}            % Anpassungen am Inhaltsverzeichnis
\usepackage{tabulary}
\contentsmargin{0.7cm}           % Abstand im Inhaltsverzeichnis zw. Punkt und Seitenzahl
\usepackage[                     % Klickbare Links (enth. auch "nameref", "url" Package)
  hidelinks,                     % Blende die "URL Boxen" aus.
  breaklinks=true                % Breche zu lange URLs am Zeilenende um
]{hyperref}
\urlstyle{same}                  % Aktuelle Schrift auch für URLs
% Anpassung von autoref für Gleichungen (ergänzt runde Klammern)
\addto\extrasngerman{%
	\def\equationautorefname~#1\null{Gleichung~(#1)\null}
}



% ---- Für das Quellenverzeichnis
\usepackage[
	backend = biber,                % Verw. von biber
	language = auto,
	style = numeric,                % Nummerierung der Quellen mit Zahlen
	sorting = none,                 % none = Sortierung nach der Erscheinung im Dokument
	block = space,                  % Extra Leerzeichen zwischen Blocks
	hyperref = true,                % Links sind klickbar auch in der Quelle
	%backref = true,                % Referenz, auf den Text an die zitierte Stelle
	bibencoding = auto,
	giveninits = true,              % Vornamen werden abgekürzt
	doi=false,                      % DOI nicht anzeigen
	isbn=false                      % ISBN nicht anzeigen
]{biblatex}
\addbibresource{Inhalt/literatur.bib}
\setcounter{biburlnumpenalty}{3000}     % Umbruchgrenze für Zahlen
\setcounter{biburlucpenalty}{6000}      % Umbruchgrenze für Großbuchstaben
\setcounter{biburllcpenalty}{9000}      % Umbruchgrenze für Kleinbuchstaben
\DeclareNameAlias{default}{last-first}  % Nachname vor dem Vornamen
\AtBeginBibliography{\renewcommand{\multinamedelim}{\addslash\space
}\renewcommand{\finalnamedelim}{\multinamedelim}}  % Semikolon zwischen den Autorennamen
\DefineBibliographyStrings{german}{
  urlseen = {Einsichtnahme:},                      % Ändern des Titels von "besucht am"
}
\usepackage[babel,german=quotes]{csquotes}         % Deutsche Anführungszeichen + Zitate


% ---- Für Mathevorlage
\usepackage{amsmath}    % Erweiterung vom Mathe-Satz
\usepackage{amssymb}    % Lädt amsfonts und weitere Symbole
\usepackage{MnSymbol}   % Für Symbole, die in amssymb nicht enthalten sind.


% ---- Für Quellcodevorlage
\usepackage{scrhack}                    % Hack zur Verw. von listings in KOMA-Script
\usepackage{listings}                   % Darstellung von Quellcode
\usepackage{xcolor}                     % Einfache Verwendung von Farben


% ---- Tabellen
\usepackage{booktabs}  % Für schönere Tabellen. Enthält neue Befehle wie \midrule
\usepackage{multirow}  % Mehrzeilige Tabellen
\usepackage{siunitx}   % Für SI Einheiten und das Ausrichten Nachkommastellen
\sisetup{loctolang=DE:ngerman, decimalsymbol=comma} % Damit ein Komma und kein Punkt verwendet wird.

% ---- Für Figuren im Text
\usepackage{wrapfig}
\usepackage{placeins}

% ---- Für Definitionsboxen in der Einleitung
\usepackage{amsthm}                     % Liefert die Grundlagen für Theoreme
\usepackage[framemethod=tikz]{mdframed} % Boxen für die Umrandung

\setlength{\parskip}{\baselineskip} % Abstand zwischen Paragraphen - Noch mehr Seiten ercheaten ;-)


% Für Zeichungen (FlowCharts, Automaten, ...)
\usepackage{tikz}
\usetikzlibrary{shapes.geometric, arrows}

\tikzstyle{startstop} = [rectangle, rounded corners, minimum width=3cm, minimum height=1cm,text centered, draw=black]
\tikzstyle{io} = [trapezium, trapezium left angle=70, trapezium right angle=110, minimum width=3cm, minimum height=1cm, text centered, draw=black]
\tikzstyle{process} = [rectangle, minimum width=3cm, minimum height=1cm, text centered, draw=black]
\tikzstyle{decision} = [diamond, minimum width=3cm, minimum height=1cm, text centered, draw=black, aspect=2]
\tikzstyle{arrow} = [thick,->,>=stealth]

% Highlight-Boxen
% ---- Grundsätzliche Definition zum Style
\newtheoremstyle{defi}
  {\topsep}         % Abstand oben
  {\topsep}         % Abstand unten
  {\normalfont}     % Schrift des Bodys
  {0pt}             % Einschub der ersten Zeile
  {\bfseries}       % Darstellung von der Schrift in der Überschrift
  {:}               % Trennzeichen zwischen Überschrift und Body
  {.5em}            % Abstand nach dem Trennzeichen zum Body Text
  {\thmname{#3}}    % Name in eckigen Klammern
\theoremstyle{defi}

% ------ Definition zum Strich vor eines Texts
\newmdtheoremenv[
  hidealllines = true,       % Rahmen komplett ausblenden
  leftline = true,           % Linie links einschalten
  innertopmargin = 8pt,      % Abstand oben
  innerbottommargin = 4pt,   % Abstand unten
  innerrightmargin = 0pt,    % Abstand rechts
  linewidth = 3pt,           % Linienbreite
  linecolor = gray!40,       % Linienfarbe
]{defStrich}{Definition}     % Name der des formats "defStrich"

\begin{document}

\newcommand{\autor}{Heidinger, Matthis, Riesinger, Stephan}
\newcommand{\kurs}{TINF17B1}
\newcommand{\titel}{4-Gewinnt auf einem Mikrocomputer der 8051-Famile}

\thispagestyle{empty}
\begin{titlepage}
\enlargethispage{4cm}

\begin{center}
  \huge{\textbf{\titel}}\\[1.5cm]
  \normalsize{von}\\[1ex] \Large{\textbf{\autor}} \\[1cm]

	\Large{Kurs \kurs}\\[0.5cm]
\end{center}

\end{titlepage}

\newpage

\tableofcontents

\chapter{Einleitung}

\section{Motivation}

Mit dem 8051 ist ein relativ einfacher Einstieg in die Assemblerprogrammierung möglich. Assembler ist sehr maschinennah und bietet aus diesem Grund
nicht die Annehmlichkeiten, die beispielsweise objektorienterte Programmiersprachen wie Java mit sich bringen.\\
An einem Simulator lässt sich gefahrlos erproben, wie mit Pointern und sehr begrenztem Speicherplatz umzugehen ist.\\
Durch die andere Herangehensweise an ansonsten vertraute Programmierstrukturen wie zum Beispiel Vergleichen, die nun mit Jump-Befehlen und negativen
Vergleichen implementiert werden müssen, wird das logische Denkvermögen geschult.

\section{Aufgabenstellung}

Es soll ein Spiel nach dem bekannten Spielkonzept von \glqq 4-Gewinnt\grqq{} in Assembler auf dem 8051 entwickelt und
implementiert werden.\\
Dafür wird für die Visualisierung der Spielfläche eine entsprechende Hardware gewählt,
auf der die Spielstände der beiden Spieler angezeigt werden (Output). Die Auswahl der Spalte, in der ein
\glqq Spielstein\grqq{} platziert werden soll, muss ebenfalls über eine entsprechende Hardware gelöst werden (Input).\\
Die beiden Spieler müssen damit in der Lage sein, abwechselnd sogenannte Spielsteine in selbst ausgewählte Spalten zu werfen,
welche dann am oberen Ende des Stapels angefügt werden. Hat ein Spieler es geschafft, 4 seiner Spielsteine hintereinander
in eine Reihe oder Spalte zu bringen, hat er gewonnen.\\
Dabei soll sicher gestellt sein, das nicht mehr Spielsteine in eine Spalte geschmissen werden als Platz ist. Auch wichtig ist, dass die Spielsteine
der beiden Spieler klar voneinander zu unterscheiden sein müssen, damit es nicht zu Verwechslungen kommen kann.

\chapter{Grundlagen}

\section{Assembler}

Assembler ist toll!

\section{Der 8051 Mikrocomputer}

Die ersten Mikroprozessoren der 8051-Reihe wurden im Jahr 1980 von Intel entwickelt. Es handelt sich dabei um einen direkten Nachfolger der 8048-Familie.
Die 8051-Familie erfreute sich extrem großer Beliebtheit. So wurden über 250 Familienmitglieder von verschiedensten Herstellern wie Philips, Siemens, AMD, OKI und weiteren gebaut und produziert. Der Höhepunkt der Beliebtheit des 8051 war das Jahr 1995, in welchem diese Mikroprozessorfamilie einen Marktanteil von bis zu 30 Prozent erreichte und täglich mehr als eine Million Prozessoren hergestellt wurden.

Auch technisch war der 8051 zu damaligen Zeiten hochmodern, was man folgenden Eckdaten entnehmen kann:
\begin{itemize}
	\item $1,2$ - $18 MHz$ Taktrate, oft werden $12 MHz$ verwendet
	\item 4 kByte ROM
	\item 128 Byte RAM
	\item 4 8-Bit Eingabe- und Ausgabeports
	\item 2 16-Bit Zähler beziehungsweise Zeitgeber
	\item Eine USART-Schnittstelle
	\item 5 Interruptquellen
	\item Bei einer Taktrate von $12 MHz$ laufen
	      \begin{itemize}
		      \item $58\%$ der Befehle in $1 \mu s$
		      \item $40\%$ der Befehle in $2 \mu s$
		      \item $2\%$ der Befehle in $4 \mu s$
	      \end{itemize}
	      ab. Die langsamsten Befehle sind beispielsweise Multiplikation und Division.
\end{itemize}


\section{Entwicklungsumgebung MCU-8051 IDE}

\textbf{LANGSAM!!!}

\chapter{Konzept}

Wir machen ein 4-Gewinnt

\section{Analyse}

Was sollen wir denn Analysieren?


\section{Programmentwurf}

% TODO: Irgendwo "eingabebereit" erklären
Der Programmfluss, wie er in \autoref{fig:programmfluss} dargestellt ist, beschreibt den Ablauf des Programms.\\
Nach dem Start des Simulators beginnt die Initialisierung des Programms - der benötigte Speicher wird zurückgesetzt, Spieler 1 als aktiver Spieler ausgewählt und der Timer gestartet. Danach beginnt die Hauptschleife.\\
In ihr wird wiederholt abgefragt, ob aktuell eine Eingabe getätigt werden kann - also, ob das Keypad vollständig auf false gestellt wurde.
Ist dies möglich prüft eine Schleife auf eine Eingabe durch den aktiven Spieler. Wurde diese ausgeführt, wird anhand des Zuges überprüft, ob der Spieler es geschafft hat 4 Steine nebeneinander zu setzen. Ist dies der Fall, hat er gewonnen - andernfalls kommt es zum Spielerwechsel und die Hauptschleife beginnt von Neuem.

\begin{figure}
	\centering
  \begin{tikzpicture}[node distance=1.5cm]
    \node (start) [startstop] {Start};
    \node (init) [process, below of=start] {Initialisierung};
    \node (mainloop) [process, below of=init] {Eingabepad prüfen};
    \node (readyforinput) [decision, below of=mainloop, yshift=-1.0cm] {Eingabe zurückgesetzt?};
    \node (getinput) [process, below of=readyforinput, yshift=-1.0cm] {Eingabepad prüfen};
    \node (checkplayerdecision) [decision, below of=getinput, yshift=-1.0cm] {Eingabe gesetzt?};
    \node (checkturn) [process, below of=checkplayerdecision, yshift=-1.0cm] {Matrix prüfen};
    \node (checkwin) [decision, below of=checkturn, yshift=-1cm] {Gewonnen?};
    \node (switchplayers) [process, right of=checkwin, xshift=5cm] {Spieler wechseln};
    \node (end) [startstop, below of=checkwin, yshift=-1cm] {Ende};
    \draw [arrow] (start) -- (init);
    \draw [arrow] (init) -- (mainloop);
    \draw [arrow] (mainloop) -- (readyforinput);
    \draw [arrow] (readyforinput) -- node[anchor=west] {ja} (getinput);
    \draw [arrow] (readyforinput) -- node[anchor=south] {nein} (readyforinput-|switchplayers.west) |- ([yshift=8pt]mainloop.south east);
    \draw [arrow] (getinput) -- (checkplayerdecision);
    \draw [arrow] (checkplayerdecision) -- node[anchor=south] {nein} (checkplayerdecision-|switchplayers.west) |- ([yshift=8pt]getinput.south east);
    \draw [arrow] (checkplayerdecision) -- node[anchor=west] {ja} (checkturn);
    \draw [arrow] (checkturn) -- (checkwin);
    \draw [arrow] (checkwin) -- node[anchor=south] {nein} (switchplayers);
    \draw [arrow] (checkwin) -- node[anchor=west] {ja} (end);
    \draw [arrow] (switchplayers) |- ([yshift=-8pt]mainloop.north east);
  \end{tikzpicture}

  \caption{Programmfluss des Spieles}
	\label{fig:programmfluss}
\end{figure}

Zusätzlich zu dieser Hauptschleife wird über ein Timerinterrupt regelmäßig eine Routine ausgeführt, welche das aktuelle Spielbrett ausgibt. Dabei werden die vom ersten Spieler gesetzten Steine auf der LED-Matrix durchgängig beleuchtet, während die Steine des zweiten Spielers blinken.

\begin{defStrich}[Hinweis]
  Die Wiederholrate, mit welcher die LED-Matrix aktualisiert wird ist zu groß, um den Blinkeffekt sehen zu können. Sie musste allerdings für die Entwicklung im Simulator so stark erhöht werden, da der Simulator deutlich langsamer als die tatsächliche Hardware ist.
\end{defStrich}

\chapter{Implementation}

In Assembler

\chapter{Zusammenfassung}

War gut, nochmal machen!


\end{document}